Advisors and Mentors are some of the most important people any scientist (or
human being) can encounter. Time and again, I have heard colleagues speak of
"that one inspirational" person that drove them to be their best, helped them
grow personally and professionally, and refused to give up on them even when
doing so might even be the 'correct' course of action.

I am very lucky in that I suffer from an embarassment of riches in this regard.
The path I've trodden over the course of my life, academic career, and
transition to a professional career has been litered with people who have
uplifted, supported, and educated me. For that, I am profoundly grateful, and
owe a debt that can truly only be paid by mentoring, supporting, and uplifting
others.

I am very greatful to my advisor, Ken Barish, whose calm, stoic and unabated
support helped guide me through my research. Ken involved me in many aspects of
the research group at UCR, beyond the scientific work. He insured that I was
exposed to all aspects of research in particle physics, including writing
grants, reviewing literature, mentoring younger students, building detectors,
running a research group.  Ken has always had the uncanny ability to know "who
to talk to", and has used that ability to connect me with other excellent
physicists, who in turn helped me grow as a researcher and human being. Ken's
style of advising gave me the freedom I needed to pursue my interests, and move
in the scientific directions I felt most fruitful, while still helping to
provide provide an overall direction for my academic career and research. 

Thought maybe it sounds a bit silly, I am most personally greatful to Ken for
when he gave me a second chance in graduate school. I suspect he had not read
my transcript, or perhaps just direly needed students in his research group,
because when I joined, I was perhaps the least skilled person in my graduate
school cohort - indeed - I received the absolute lowest passing grade on the
dreaded comprehensive exams - any lower, and my graduate career would have
ended in short order.

When Ken accepted me into his group, I was an undoubtedly risky choice. I
struggled mightily my first year in grad school.  I earned poor grades and had
effectively failed the first course in Electricity and Magnetism. In fact, my
performance was so poor, that the administration reduced my teaching
privilags to give me more time to focus on academics (though, I can happily
report that I am an award winning teacher).

Eventually, I lost my graduate division fellowship, which ultimately meant that
I had no income, no means of supporting myself. I was effectively dismissed from
graduate school. 

However, I was interested in the research carried out by Ken and Rich Seto's
heavy ion group, so I talked to Ken, who graciously accepted me into the group,
provided me with academic and financial support, and even flew me out to
Brookhaven National Lab my first summer of graduate school. I finally got to
dive into 'real' physics research. I think it was this vote of confidence from
Ken, as well as the awesome physics happening at the PHENIX experiment which
gave me the confidence to wholeheartedly devote myself to my studies and
research, and ultimately succeed, in the face of imminent failure. Without Ken's
intervention, I fear that my graduate career would have been over in short
order.

While at Brookhaven National Lab, I encountered graduate students, post docs,
research staff, and other amazing physicists who taught me an incredible
amount. I was met with patience, kindess, friendship and mentorship.

Richard Hollis was one of the first people I encoutered in my research group at
UCR - I have never met a more patient person. Richard helped me get my bearings,
and set me straight, during my early (and later) years of graduate school. Oleg
Eyser was with our group at that time as well - although I recall that he was
less than thrilled to have yet another green graduate student constantly asking
questions, taking time away from his work. He still made time to teach me, and
introduced me to the very complicated PHENIX software system.  Oleg challenged
me to find answers for myself, and was unrelenting in that regard. I am certain
that this made me a better researcher.

Josh Perry gave me a crash course on the PHENIX data acquisition system,
boiling down this incredibly complicated system into understandable pieces, and
helped me learn that ultimately, persistance pays off when tackling difficult
problems. Martin Leitgab took me under his wing while I worked days and nights
to learn PHENIX's fast data production systems. Martin's systematic, calm, and
patient approach to problem solving has been something I have tried to emulate
since my work with him - I could not have asked for a better mentor for that
project. On that same project was my first introduction to Chris Pinkenburg and
Martin Purschke - somewhat of the yin and yang of the PHENIX online data
aquisition. I benefited enormously from conversations with both about PHENIX
software, and online systems. Martin Purschke's kindness and sense of humor
always spurred me on, while Chis' dogged dedication to doing things 'the right
way' kept me honest. I have returned to Martin with various questions many
times over the years, and he has always been cheerful, supportive and wise with
his answers. 

Likely no one has been woken up so many times with emergencies at the PHENIX
counting house in the middle of the night then Martin, yet even when I woke him
at 3 am on many occasions, would simply state, in an execptionally dry, well
practiced line: 'Martin Speaking, please state the nature of your emergency'. I
don't know of many who can manage to be coy and good natured under such
circumstances. I recall on another occasion, I was the one recieving a late
night call - and I did my best to channel Martin when I was greeted from the
PHENIX counting house with: "Mike, what the hell is wrong with the spin
montor!?".

I have to acknoledge Joe Seele as well, in this regard, as he probably more
than anyone else, set me on the path to learning to program well, and using a
computer effectively - these skills, so often neglected in Particle Physics,
have paid off for me, many, many times over.

There is no way that I would have grown as much as I did in my later years in
grad school, had I not had the fabulous opportunity to work with Ralf Seidl,
Francesca Giordano, and Daniel Jumper. Francesca tutored me in the ways of
hardware assembly - it was quite a proud moment for me to see the detectors
that I put together with my own hands, mounted on the front of the PHENIX muon
tracker nose cone. Francesca knocked it out of the park with her leaderhship in
the RPC program. She has this ineffable dogged determination, and ability to
break down problems into manageable pieces - a quality I have done my best to
emulate, and see in nearly all physicsists that I respect.

I owe Ralf Seidl a huge debt for his persistance and leadership in the
W-Analysis, which to date has generated (or will generate) at least six
different theses. Ralf has always demonstrated a willingness to walk students
through a tough problem, and an admirable hard-headedness when it comes to
dealing without people that maybe need to put in a bit more effort in their
research and work. Though I have at times been on the recieving end of this, the
result has always been a very strong motivation to do right by him, and pick up
my slack.

Daniel has been an absolute lifesaver, and a dear friend. He and I have worked
essentially in concert on the W-Analysis, and our frequent collaboration over
skype, in person, and around the world has really helped keep me honest.  We
often traded the latest VIM secrets, and carefully cross-checked eachothers'
work. It is rare to have a colleauge like Daniel backing you up, and I'm very
grateful.

I am also greatful for the rest of the W-Analysis team, Abraham Meles, Sangwha
Park and Chong Kim. Everyone has doggedly stuck together for many cross-checks,
and when our analysis is finally published, the collective hard work of Ralf,
Francesca, Sanghwa, Chong and Abraham will shine through.

Friends and Family \\
  Bob Beaumier, 
  Marian Beaumier, 
  Joe Beaumier, 
  David Beaumier, 
  Emily Vance, 
  Jackie Hubbard, 
  Alexander Anderson-Natalie, 
  Corey Kownacki, 
  Chris Heidt, 
  Pat Odenthal, 
  Behnam Darvish Sarvestani, 
  Oleg Martynov, 
