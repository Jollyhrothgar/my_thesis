\chapter{Introduction}

\textbf{\textcolor{red}{THIS THESIS IS CURRENTLY AN UNPUBLISHED DRAFT. IT HAS
NOT BEEN SUBMITTED AND MAY HAVE MISSING CITATIONS AND CONTENT}}

A brief note - figures used here \textit{without} attribution were either:
produced by me, produced in collaboration with others in my working group, or
obtained from authors who labeled them for reuse without attribution. I have taken
great pains to cite all figures, tables, and content; even if taken from a
source which does not require attribution. Happily, any work undertaken in the
United States that is the beneficiary of Federal funding is
\textit{automatically} public domain, and may be reused \textit{without
attribution}. This naturally includes work produced by any experiment receiving
DOE funding. Of course, anyone who makes a habit of reusing another's hard work
under the protection of this particular facet of copyright law, might want to
carefully reflect on the various ways of translating \textit{'teste di cazzo'}
from Italian to English.

\section{A Brief History of the Proton}
The angular momentum of the proton and neutron has been a subject of study for
the last 30 years. One of the challenges of particle physics is to
create a framework which can accurately describe matter, as well as predict the
behavior of matter at all energy scales. Protons and neutrons are baryons which
make up the majority of the mass in the visible universe, yet fully
understanding the origins of their properties - such as  mass and spin, still
eludes us. However, through the application of the scientific method over many
generations of physicists, we have magnificently described this important
particle, and understood much of its properties. However, one property which
still defies our descriptions is its fundamental angular momentum, spin. \\
	
Our understanding of the proton has evolved and sharpened since the first
experiments in deep inelastic scattering showed that the proton is not a
fundamental particle~\cite{Breidenbach1969}. Gell-Mann later planted the seeds
of a theoretical framework which could in part describe some of the structure of
baryons, a class of hadrons which we may naively describe as composed of three
'valence quarks'~\cite{Bjorken1969}. We can apply well known spin-sum rules to the
individual spins of the valence quarks which compose the proton in our naive
valence-model to produce a correct prediction for the protons' spin
${1}\over{2}$. When experimenters set out to measure the contribution of these
valence quarks in 1988 at the EMC experiment \cite{Ashman1988}, they were
flabbergasted to find that the valence quarks carry only a small fraction of the
proton's spin. Although recent papers \cite{Povh2016} suggest that this 'spin
crisis' is simple due to mis-attribution of spin, most literature to date has
focused on understanding how to model the proton with parton distribution
functions. These parton distribution functions come in many varieties, and probe
different degrees of freedom within the proton, in both the case of unpolarized
parton distribution functions, and polarized parton distribution functions. \\
 
\section{Scope and Objectives of This Work} This thesis will describe the
research I carried out between May of 2010 through August of 2016. I will often
quote work that was carried out in active collaboration with Ralf Seidel,
Francesca Giordano, Daniel Jumper, Sanghwa Park, Abraham Meles and Chong Kim.
Daniel, Abraham, Ralf, Francesca, and myself all worked on the 2013 polarized
proton data set taken at RHIC with PHENIX. This analysis comprises the body of
work devoted to calculating $A_L$ for the $W\rightarrow\mu$ decay. Since 2013,
the five of us collaborated closely on all aspects of the work, which provided
invaluable cross-checks at nearly every stage. Many of the figures in this
document were produced by our collective efforts, and I will do my best to cite
when possible, if one analyzer played a particularly large role in generating
the data or visualization, however after several years of working together, I
will certainly fail to attribute, or mis-attribute at times.

The other portion of this thesis will discuss the Vernier Analysis, which is
instrumental for every single-cross-section calculation taken with RHIC data.
The thrust of the Vernier Analysis is to determine the beam luminosity at
PHENIX's interaction point, so as to normalize these cross-section calculations.
This is done with a series of specialized Vernier-Scans, where beams are scanned
across one-another in order to measure beam geometry. The luminosity can then be
calculated from first principals, and compared to the advertised machine
luminosity published by RHIC's collider-accelerator department. I began working
with the Vernier Analysis under the tutelage of K. Oleg Eyser, but eventually
moved to work independently on the analysis, producing an entire software
framework for handling data cleaning, analysis, visualization and simulation.

As an additional note, while I attempt to be consistent with my notation, my
convention, when pulling mathematics from cited sources, is to keep the source
mathematical notation. Additionally, for other sections, when I reproduce
definitions or calculations which are perhaps more related to general PHENIX
analyses, I will attempt to emulate the notational style found in Hideyuki
Oide's thesis (~\cite{Oide2012}), which has served as a seminal document for
guiding this analysis, as well as the Run 12 analysis.
