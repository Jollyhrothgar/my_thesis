\chapter{Introduction}

\textbf{\textcolor{red}{THIS THESIS IS CURRENTLY AN UNPUBLISHED DRAFT. IT HAS
NOT BEEN SUBMITTED AND MAY HAVE MISSING CITATIONS AND CONTENT}}

The angular momentum of the proton and neutron has been a subject of study for
the last 30 years, intensifying after the finding in the late 1980's that a
three-valence quark model does not accurately predict the spin of the proton.
One of the challenges of particle physics is to create a framework which can
accurately describe matter, as well as predict the behavior of matter at all
energy scales. Protons and neutrons are baryons which make up the majority of
the mass in the visible universe, yet fully understanding the origins of their
properties - such as  mass and spin, still eludes us. However, through the
application of the scientific method over many generations of physicists, we
have magnificently described this important particle, and understood much of its
properties. However, one property which still defies our descriptions is its
fundamental angular momentum, spin.
	
Our understanding of the proton has evolved and sharpened since the first
experiments in deep inelastic scattering showed that the proton is not a
fundamental particle~\cite{Breidenbach1969}. Gell-Mann later planted the seeds
of a theoretical framework which could in part describe some of the structure of
baryons, a class of hadrons which we may n{\"a}ively describe as composed of three
`valence quarks'~\cite{Bjorken1969}. We can apply well known spin-sum rules to the
individual spins of the valence quarks which compose the proton in our naive
valence-model to produce a correct prediction for the proton's spin
${1}\over{2}$. When experimenters set out to measure the contribution of these
valence quarks in 1988 at the EMC experiment \cite{Ashman1988}, they were
surprised to find that the valence quarks carry only a small fraction of the
proton's spin, especially in light of the fact that in the three-quark model,
one can easily build a spin $1/2$ particle from three spin $1/2$ quarks. 

Although recent papers \cite{Povh2016} suggest that this `spin
crisis' is simple due to mis-attribution of spin, most literature to date has
focused on understanding how to model the proton with parton distribution
functions. These parton distribution functions come in many varieties, and probe
different degrees of freedom within the proton, in both the case of unpolarized
parton distribution functions, and polarized parton distribution functions. 
 
\section{Scope and Objectives of This Work} 

In the first part of this thesis, I will describe the research I carried out
between May of 2010 through August of 2016.  This analysis comprises the body of
work devoted to calculating $A_L$ for the $W\rightarrow\mu$ decay. The results
of this analysis are used in global fits to constrain the total contribution of
quarks and anti-quarks in the so-called `proton-sea' to the proton's total spin.

In the second portion of this work, I will discuss the `Vernier Analysis', which
is instrumental for every single-cross-section calculation taken with RHIC data.
The thrust of the Vernier Analysis is to determine the beam luminosity at
PHENIX's interaction point. This enables one to normalize the results to the p+p
cross-section. This is done with a series of specialized Vernier-Scans, where
beams are scanned across one-another in order to measure beam geometry. The
luminosity can then be calculated from first principals, and compared to the
estimated machine luminosity published by RHIC's collider-accelerator
department. I produced an entire software framework for handling data cleaning,
analysis, visualization and simulation.
