Advisors and Mentors are some of the most important people any scientist will
encounter in their professional career. Time and again, I have heard colleagues
speak of "that one inspirational" person that drove them to be their best, and
knew how to "grow" a researcher. 

I am very greatful to my advisor, Ken Barish, whose calm, stoic and unabated
support helped guide me through my research. Ken involved me in many aspects of
the research group at UCR, beyond the scientific work. He insured that I was
exposed to all aspects of research in particle physics, including writing
grants, reviewing literature, mentoring younger students, building detectors,
running a particle accelelrator detector, and of course, data analysis.  Ken has
always had the uncanny ability to know "who to talk to" for nearly any problem I
might have. Ken connected me with other excellent physicists, who helped me grow
as a researcher, and he gave me the freedom I needed to pursue my interests, and
move in the scientific directions I felt most fruitful, while helping to provide
an overall direction for my academic career and research. 

Beyond all this, the single most important thing Ken has done for me, is to give
me a second chance in graduate school. When he accepted me into his group, I was
an undoubtedly risky choice. I struggled mightily my first year in grad school.
I earned poor grades, and even had to re-take a class. In fact, my performance
was so poor, that my teaching responsibilities were reduced, and eventually, I
lost my graduate division fellowship, which ultimately meant that I had no
income, or means of supporting myself; I was effectively dismissed from graduate
school. However, I was interested in the research carried out by Ken and Rich
Seto's heavy ion group, so I talked to Ken, who graciously accepted me into the
group, provided me with academic and financial support, and even flew me out to
Brookhaven National Lab my first summer of graduate school. I finally got to
dive into 'real' physics research. I think it was this vote of confidence from
Ken, as well as the awesome physics happening at the PHENIX experiment which
gave me the confidence to wholeheartedly devote myself to my studies and
research. Without Ken's vote of confidence, I fear that my graduate career would
have been over in short order.

While at Brookhaven National Lab, I encountered graduate students, post docs,
research staff, and other amazing physicists who taught me an incredible
amount, and showed both patience, kindess, friendship and mentorship to me.
Richard Hollis was one of the first people I encoutered in my research group at
UCR - I have never met a more patient person. Richard helped me get my
bearings, and set me straight, during my early (and later) years of graduate
school. Oleg Eyser was with our group at that time as well - although I recall
that he was less than thrilled to have yet another green graduate student
constantly asking questions, taking time away from his work. He still made time
to teach me, and introduced me to the very complicated PHENIX software system.
Oleg challenged me, and expected me to find answers for myself, and was
unrelenting in that regard, which I am certain made me a better researcher.

Josh Perry gave me a crash course on the PHENIX data acquisition system,
boiling down this incredibly complicated system into understandable pieces, and
helped me learn that ultimately, persistance pays off when tackling difficult
problems. Martin Leitgab took me under his wing while I worked days and nights
to learn PHENIX's fast data production systems. Martin's systematic, calm, and
patient approach to problem solving has been something I have tried to emulate
since my work with him - I could not have asked for a better mentor for that
project. On that same project was my first introduction to Chris Pinkenburg and
Martin Purschke - somewhat of the yin and yang of the PHENIX online data
aquisition. I benefited enormously from conversations with both about PHENIX
software, and online systems. Martin Purschke's kindness and sense of humor
always spurred me on, while Chis' dogged dedication to doing things 'the right
way' kept me honest. I have returned to Martin with various questions many
times over the years, and he has always been cheerful, supportive and wise with
his answers. Probably nobody other than Ed Desmond has been woken up so many
times with emergencies at the PHENIX counting house in the middle of the night,
yet even when I woke him at 3 am on many occasions, would simply state, in an
execptionally dry, well practiced line: 'Martin Speaking, please state the
nature of your emergency'. I don't know of many who can manage to be coy and
good natured under such circumstances.

I have to acknoledge Joe Seele as well, in this regard, as he probably more
than anyone else, set me on the path to learning to program well, and using a
computer effectively - these skills, so often neglected in Particle Physics,
have paid off for me, many, many times over.

W Analysis Crew \\
  Ralf Seidl, 
  Francesca Giordano, 
  Sangwha Park, 
  Daniel Jumper, 
  Abraham Meles,
  Chong Kim, 

Friends and Family \\
  Bob Beaumier, 
  Marian Beaumier, 
  Joe Beaumier, 
  David Beaumier, 
  Emily Vance, 
  Jackie Hubbard, 
  Alexander Anderson-Natalie, 
  Corey Kownacki, 
  Chris Heidt, 
  Pat Odenthal, 
  Behnam Darvish Sarvestani, 
  Oleg Martynov, 
