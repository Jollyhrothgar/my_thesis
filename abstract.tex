\begin{abstract}
  This thesis discusses the process of extracting the longitudinal asymmetry,
  $A_L^{W\pm}$, describing $W\rightarrow\mu$ production in forward kinematic
  regimes. This asymmetry is used to constrain our understanding of the
  polarized parton distribution functions characterizing $\bar{u}$ and $\bar{d}$
  sea quarks in the proton. This asymmetry will be used to constrain the overall
  contribution of the sea-quarks to the total proton spin. The asymmetry is
  evaluated over the pseudorapidity range of the PHENIX Muon Arms, $2.1 <
  |\eta|2.6$, for longitudinally polarized proton-proton collisions at 510 GeV
  $\sqrt{s}$. In particular, I will discuss the statistical methods used to
  characterize real muonic $W$ decays and the various background processes is
  presented, including a discussion of likelihood event selection and the
  Extended Unbinned Maximum Likelihood fit. These statistical methods serve
  estimate the yields of $W$ muonic decays, which are used to calculate the
  longitudinal asymmetry.
\end{abstract}
