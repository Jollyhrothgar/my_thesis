\chapter{Models and Associated Probes For Proton Spin Structure}
\label{ch:modeling_proton_spin}

With the advances made over the last half-century, we have come very close to
obtaining a complete model describing the world around us. In the realm of
Quantum Chromodynamics, there is great difficulty in writing down a model for
sub-nuclear structure, due to our inability to directly image the dynamics of
matter on this scale. Though progress is being made with the newest plans for
electron-ion scattering, we do not yet have a coherent three dimensional picture
of the structure of protons and neutrons - the bulk of the visible mass in the
universe. 

One major hurdle is that with our current experimental devices and
theoretical understanding of proton structure, we lack direct probes. We must
often deal with fragmentation and non-perturbative effects - which are
notoriously difficult to include in analytical models. Additionally, we find
that the very structure and distribution of partons and gluons in the nucleus is
a scale-dependent phenomena, that is to say, if we take measurements at a lower
energy, we get a different distribution of partons and gluons than if we measure
at higher energy. This is not to say that the proton magically changes itself
based on energy, but is really more related to the length scale that we are
probing inside of the proton. With higher energies, we probe successively
shorter length scales. Therefore, to form a complete picture of proton spin
structure, we must probe by scanning over a broad range of energies (length
scales).

This leaves quite a rich theoretical and experimental landscape to explore,
since many experiments are configured to probe a few energy scales, if not just
one energy scale. Additionally, some probes may be available at one experiment
(i.e. an electron scattering fixed target experiment) but not other
experiments (i.e. a collider experiment).

In the previous chapter, we discussed in excruciating detail the history behind
studying the structure of matter, leading up to a brief discussion of the
contemporary experiments in proton spin structure. Glaringly, I neglected to
discuss the Relativistic Heavy Ion Collider (RHIC) and the Pioneering High
Energy Nuclear Interaction eXperiment (PHENIX), since I wanted to put the
program into a firm theoretical context in this chapter.

RHIC is a very special accelerator in that it can vary the collisions energy
over a substantially large energy range (7.7 $GeV$ - 510 $GeV$ $\sqrt{s}$).
Many other experiments have covered kinematic ranges that could be explored at
RHIC, and in cases where greater precision or statistics are available, RHIC has
been exploited to do so.

However, this thesis will discuss the experimental efforts of PHENIX to do
something no other experiment has done - utilize the production of W-Bosons as a
direct probe of proton spin. Though W-Bosons obviously can be created in
collisions with the right ingredients and correct energy, the W-Bosons that
we're interested in at RHIC are very special. The collision conditions around
the protons at colliding at PHENIX provides just enough energy to create real
W-Bosons from interaction of quarks and anti-quarks between two colliding
protons. The energy is not sufficiently high enough to produce real W-Bosons
from other processes in amounts which would significantly dilute the primary
source.

\textbf{Questions I'd Like to Answer in this Chapter}
\begin{enumerate}
    \item Why can we collide two polarized protons, but pretend that only one of
      them is polarized? What if summing over polarization states doesn't
      'cancel' this out?
    \item Why is $A_L$ an appropriate probe for proton spin
    \item How exactly do we go from a measurement of $A_L$ to an understanding
      of proton spin?
    \item Why do we need to calculate $A_{LL}$? What does it tell us? Why do we
      combine the helicities in the way we do, to define $A_{LL}$?
\end{enumerate}

Double spin asymmetry:

\begin{equation}
  A_{LL} = {
    {d\sigma^{\Rightarrow\Rightarrow}-d\sigma^{\Leftarrow\Rightarrow}}
    \over
    {d\sigma^{\Rightarrow\Rightarrow}+d\sigma^{\Leftarrow\Rightarrow}}
  }
\end{equation}

Although Gell-Mann's simple quark model of baryons  \cite{Gell-Mann1961}
predicts the correct quantity for the spin of the proton, the work of Ashman et
al (1988)~\cite{Ashman1988} at the European Muon Collaboration directly measured a
portion of the proton structure function $g_1$ and found that a rather small
fraction of the proton spin comes from quarks - and most of the spin is carried
by the gluons (Figure~\ref{fig:emc_g1_result}).

\begin{figure}[ht]
	\begin{center}
		\includegraphics[width=0.5\linewidth]{./figures/filler/squareimg.png}
		\caption{
			 \needfig{}  \needcap{}. Results of EMC experiment showing that the
			structure function g1, tells us a thing about proton spin.
		}
		\label{fig:emc_g1_result}
	\end{center}
\end{figure}

\section{Modeling the Proton Structure}

\begin{figure}[ht]
  \centering
  \includegraphics[width=\linewidth]{./figures/pdf_distributions_invariants.png}
  \caption{
    Figure from \cite{Accardi2012}.
  }
  \label{fig:invariants_observables_pdf}
\end{figure}

\section{Structure Functions}
Spin structure overall: \cite{Accardi2012} pp 29-31
Longitudinal spin structure: \cite{Accardi2012} pp 32-43

\begin{figure}[ht]
  \centering
  \includegraphics[width=\linewidth]{./figures/leading_twist_polarization_config.png}
  \caption{
    Figure from \cite{Accardi2012}.
  }
  \label{fig:leadings_twist_probes}
\end{figure}

\subsection{Parton Distribution Functions}

\begin{figure}[ht]
  \centering
  \includegraphics[width=0.5\linewidth]{./figures/quark_kinematics.png}
  \caption{
    Shown: quark kinematics \needcap{} 
  }
  \label{fig:quark_kinematics}
\end{figure}

\subsection{Polarized Parton Distribution Functions}
\label{sec:polarized_pdfs}
Discuss DSSV fits
\subsection{ Proton Spin Decomposition with the Ellis-Jeffe Sum Rule }

{\noindent}Gauge invariant Ellis-Jeffe
\begin{equation}
  \braket{P,{1\over2}|\hat{J_z}|P,{1\over2}}  
 = {1\over2} = {{1\over2}\Delta \Sigma +L_q+J_g}
\label{eq:ellis_jeffe_sum}
\end{equation}

{\noindent}Infinite momentum decomposition:
\begin{equation}
  \braket{P,{1\over2}|\hat{J_z}|P,{1\over2}}  
  = {1\over2} = {{1\over2}\Delta \Sigma +L_q+\Delta g + L_g}
  \label{eq:infmom_ellis_jeffe_sum}
\end{equation}

{\noindent}Quark decomposition:
\begin{equation}
  {\Delta \Sigma} =
  {
    (\Delta u+\Delta \bar{u})
    +(\Delta d + \Delta \bar{d})
    +(\Delta s + \Delta \bar{s})
  }
  \label{eq:quark_spin_decomposition}
\end{equation}

\subsection{ The Spin Asymmetry: An Experimental Probe }
Write in terms of the cross-section of polarized scattering.
\section{ that sweet table from Delia hasch}

\section{Experimental Probes for Proton Spin Structure}
\subsection{Physics Probes for the Proton Spin}

\begin{figure}[ht]
  \centering
  \includegraphics[width=\linewidth]{./figures/spin_probes.jpg}
  \caption{
    A summary of the various probes for longitudinally polarized protons. The
    \textbf{"Reaction"} column summarizes the reaction observed experimentally.
    The \textbf{"Dom. partonic process"} column describes the dominant process
    at the partonic level. The \textbf{"probes"} column shows which proton spin
    structure can be measured with the reaction. Finally, the leading order
    Feynman diagram for the partonic process is drawn. Figure is reproduced
    from: \cite{Aidala2005}.
  }
  \label{fig:spin_probes_masterspin}

\end{figure}

\subsection{W Production}

The standard model tells us that W production occurs through a pure vector-axial
interaction, this implies that the helicity of the parents particles - in
particular $u+\bar{d}\rightarrow W^+$ and $\bar{u}+d\rightarrow W^-$ have fixed
helicities, due to the relativistic final state neutrino (which is not measured,
of course). To visualize the leading order of W production, with regards to the
quark-sea element being probed, the leading order diagrams for the interaction
are shown in Figure~\ref{fig:w_probe_leading_order}~\cite{Aidala2005}

Since $\Delta q$, the polarized parton distribution function can be split into
contributions from valence quarks, and also sea quarks, understanding $\Delta
\bar{q}$ is an important step towards understanding $\Delta q$ better to better
understand the total proton spin.

\begin{figure}[ht]
  \centering
  \begin{subfigure}[b]{\textwidth}
    \centering
    \includegraphics[width=0.8\linewidth]{./figures/w_plus_u_probe.jpg}
    \caption{
      Probe for $\Delta u$ at lowest order.
    }
    \label{fig:u_probe}
  \end{subfigure}
  \begin{subfigure}[t]{\textwidth}
    \centering
    \includegraphics[width=0.8\linewidth]{./figures/w_plus_dbar_probe.jpg}
    \caption{
      Probe for $\Delta\bar{d}$ at lowest order
    }
    \label{fig:dbar_probe}
  \end{subfigure}
  \caption{
    Real $W^+$ production as produced at PHENIX. The helicity of the initial
    state fixes the helicity of the partonic participants due to the
    relativistic final state of the neutrino + the handedness of the W boson.
    $x_1$ and $x_2$ are the momentum fractions of the quarks participating from
    the participant partons~\cite{Aidala2005}. 
  }
  \label{fig:w_probe_leading_order}
\end{figure}

Though both protons in the collision are polarized, the polarization of one
participant proton can be effectively ignored by summing over all polarization
states for one of the two protons. With this assumption, we may construct a
single spin asymmetry for colliding protons by counting difference in the number
of positively and negatively polarized W's produced in collisions, scaled by the
total production:

\begin{equation}
  {{A_L}^W} =
  {{{1}\over{P}}\times{{N_{-}(W)-N_{+}(W)}\over{{N_{-}(W)+N_{+}(W)}}} }
  \label{eq:w_production_asymmetry}
\end{equation}

This is a relatively easy experimental probe to measure (assuming that we can
accurately count events which produced a W, which naturally, is nearly
impossible, as we will see in Section~\ref{sec:sbr}).

As we saw earlier, in Section~\ref{sec:polarized_pdfs}, we can write an
asymmetry in terms of the scattering cross section for the process responsible
for particle yields. These cross-sections were shown to be written in terms of
polarized parton distribution functions, thus, we cut to the chase to write down
the full expression of the theoretical asymmetries for this process in terms of
those parton distribution functions.

The following equations all contain an implied integration over $x_1$ and $x_2$.

For $W^+$ and $u$:
\begin{equation}
  {A_L^{W^+}} = 
  {
    {u_-^-(x_1)\bar{d}(x_2)-u_+^-(x_1)\bar{d}(x_2)}
    \over
    {u_-^-(x_1)\bar{d}(x_2)-u_+^-(x_1)\bar{d}(x_2)}
  }  
  \label{eq:al_u_full}
\end{equation}

For $W^+$ and $\bar{d}$
\begin{equation}
  {A_L^{W^+}} = 
  {
    {\bar{d}_-^+(x_1)u(x_2)-\bar{d}_+^+(x_1)u(x_2)}
    \over
    {\bar{d}_-^+(x_1)u(x_2)+\bar{d}_+^+(x_1)u(x_2)}
  }  
  \label{eq:al_dbar_full}
\end{equation}

Observationally, we see a superposition of \ref{eq:al_u_full} and
\ref{eq:al_dbar_full}, which is expressed in
Equation~\ref{eq:al_superposition_pos}:

\begin{equation}
  {A_L^{W^+}} = 
  {
    {
      \Delta u(x_1)\bar{d}(x_2)-\Delta \bar{d}(x_1)u(x_2)
    }
    \over
    {
      u(x_1)\bar d(x_2)+\bar(d)(x_1)u(x_2)
    }
  }
  \label{eq:al_superposition_pos}
\end{equation}

For the case of $W^-$, we observe $\bar{d}$ and $u$:
For $W^-$ and $d$:
\begin{equation}
  {A_L^{W^+}} = 
  {
    {d_-^-(x_1)\bar{u}(x_2)-d_+^-(x_1)\bar{u}(x_2)}
    \over
    {d_-^-(x_1)\bar{u}(x_2)-d_+^-(x_1)\bar{u}(x_2)}
  }  
  \label{eq:al_d_full}
\end{equation}

For $W^-$ and $\bar{u}$
\begin{equation}
  {A_L^{W^+}} = 
  {
    {\bar{u}_-^+(x_1)d(x_2)-\bar{u}_+^+(x_1)d(x_2)}
    \over
    {\bar{u}_-^+(x_1)d(x_2)+\bar{u}_+^+(x_1)d(x_2)}
  }  
  \label{eq:al_ubar_full}
\end{equation}

Observationally, we see a superposition of \ref{eq:al_d_full} and
\ref{eq:al_ubar_full}, which is expressed in
Equation~\ref{eq:al_superposition_neg}:

\begin{equation}
  {A_L^{W^-}} = 
  {
    {
      \Delta d(x_1)\bar{u}(x_2)-\Delta \bar{u}(x_1)d(x_2)
    }
    \over
    {
      d(x_1)\bar u(x_2)+\bar(u)(x_1)d(x_2)
    }
  }
  \label{eq:al_superposition_neg}
\end{equation}

Kinematics of the collision can simplify the equations even further, when at
very forward or very backward rapidities~\cite{Aidala2005}. Concretely, this is
shown via integration over the momentum fractions, $x_1$ and $x_2$, explicitly
writing the W decay in terms of the scattering cross section for polarized
proton collisions (a derivation reproduced from Hideyuki Oide's
thesis~\cite{Oide2012}):

\begin{multline}
  {
    d\sigma
    \left(
      p^{\Rightarrow}+p\rightarrow W^+\rightarrow \ell+\nu_{\ell}
    \right)
  } 
  = \\
  {
    {K\over3}\int dx_1dx_2\sum_{i,j}
    \left(
    q_{i-}^\Rightarrow(x_1)\bar{q}_{j+}(x_2) +
    \bar{q}_{j+}^\Rightarrow(x_1)q_{i-}(x_2)
    \right)
  }  \\
  \times
  {
    d\hat{\sigma}(q_i+\bar{q}_j\rightarrow W^+\rightarrow \ell^+ + \nu_{\ell})
  }
\end{multline}

{\noindent}Similarly, we may write the interaction cross-section for the
opposite helicity in the initial state:

\begin{multline}
  {
    d\sigma
    \left(
      p^{\Leftarrow}+p\rightarrow W^+\rightarrow \ell+\nu_{\ell}
    \right)
  } 
  = \\
  {
    {K\over3}\int dx_1dx_2\sum_{i,j}
    \left(
    q_{i-}^\Leftarrow(x_1)\bar{q}_{j+}(x_2) +
    \bar{q}_{j+}^\Leftarrow(x_1)q_{i-}(x_2)
    \right)
  }  \\
  \times
  {
    d\hat{\sigma}(q_i+\bar{q}_j\rightarrow W^+\rightarrow \ell^+ + \nu_{\ell})
  }
\end{multline}

Neglecting quark mass, we can assume that the helicity state of the quarks is
identical to the chirality state. Then, we substitute in the definition for
polarized parton distribution functions $\Delta q \equiv q_{+}^{\Rightarrow} -
q_{-}^{\Rightarrow}$, and sum over quark flavors, neglecting strange
contributions:

\begin{align}\label{eq:al_theory_quarks}
  {
    A_L
    \left(
      p^{\Rightarrow}+p\rightarrow W^+ \rightarrow \ell^+ +\nu_{\ell}
    \right)
  } &=  
  {
    {
      \int dx_1 dx_2 \sum_{i,j}
      \left(
        -\Delta q_i(x_1)\bar{q}_j(x_2)
        +\Delta \bar{q}_j(x_1)q_i(x_2)
      \right)\cdot d \hat{\sigma}
    }
    \over
    {
      \int dx_1 dx_2
      \sum_{i,j}(q_i(x_1)\bar{q}_j(x_2)+\bar{q}_j(x_1)q_i(x_2))\cdot d\hat{\sigma}
    }
 }\\
 & \approx  \nonumber
 {
   {
      \int dx_1 dx_2 
      \left(
        -\Delta u(x_1)\bar{d}(x_2)
        +\Delta \bar{d}(x_1)u(x_2)
      \right)\cdot d \hat{\sigma}
   }
   \over
   {
      \int dx_1 dx_2 (u(x_1)\bar{d}(x_2)+\bar{d}_j(x_1)u(x_2))\cdot d\hat{\sigma}
   }
 }
\end{align}

Since we have restricted ourselves to only the case for $u\bar{d}$, we are of
course looking at the case of $A_L^{W+}$. We may rewrite
Equation~\ref{eq:al_theory_quarks} to reflect its rapidity dependance:

\begin{equation}
  {A_L^{W+}(y_{\ell})} = 
  {
    {
     \int dx_1 dx_2 
     \left(
       -\Delta u(x_1)\bar{d}(x_2)(1-cos\hat{\theta})^2
       +\Delta \bar{d}(x_1)u(x_2)(1+cos\hat{\theta})^2
     \right)
    }
    \over
    {
       \int dx_1 dx_2 
       \left(
       (u(x_1)\bar{d}(x_2)   (1-cos\hat{\theta})^2
      +\bar{d}_j(x_1)u(x_2)) (1+cos\hat{\theta})^2
        \right)
    }
  }
  \label{eq:al_w_pos_rapidity_dependance}
\end{equation}

In this case, we follow Dr. Oide's convention of redefining $\hat{\theta}$ in
terms of the angle between the direcion of momentum of the polarized proton and
the leptop in the center of mass frame. Therefore we see kinematic isolation of
the polarized pdfs at forward or backward rapdity.\\

{\noindent}We may write $A_L^{W-}(y_{\ell})$ similarly:

\begin{equation}
  {A_L^{W-}(y_{\ell})} = 
  {
    {
     \int dx_1 dx_2 
     \left(
       -\Delta \bar{u}(x_1)d(x_2)(1-cos\hat{\theta})^2
       +\Delta d(x_1)\bar{u}(x_2)(1+cos\hat{\theta})^2
     \right)
    }
    \over
    {
       \int dx_1 dx_2 
       \left(
         (\bar{u}(x_1)d(x_2)   (1-cos\hat{\theta})^2
         +d_j(x_1)\bar{u}(x_2)) (1+cos\hat{\theta})^2
        \right)
    }
  }
  \label{eq:al_w_neg_rapidity_dependance}
\end{equation}
\clearpage
\section{Cross Sections and Luminosity}
\begin{itemize}
		\item vernier analysis note intro, equations
		\item summarize the papers on Lumoninosity
\end{itemize}

\clearpage
