\chapter{Introduction}

\section{A Brief History of the Proton}
The angular momentum of the proton has been a subject of study for the last 20
years\needcite{}. One of the challenges of particle physics is to create a
framework which can accurately describe matter, as well as predict the behavior
of matter at all energy scales. The proton is a baryon which makes up the
majority of the mass in the visible universe, yet fully understanding the
origins of its properties - such as its mass and spin, still eludes us. However,
through the applicaiton of the scientific method over many generations of
physicists, we have magnificently described this important particle, and
understood much of its properties. However, one property which still defies our
descriptions is its fundamental angular momentum, spin. \\
	
Our understanding of the proton has evolved and sharpened since the first
experiments in deep inelastic scattering showed that the proton is not a
fundamental particle~\cite{Breidenbach1969}. Gell-Mann later planted the
seeds of a theoretical framework which could in part describe some of the
structure of baryons, a class of hadrons which we may naively describe as
composed of three 'valence quarks'\needcite{}. We can apply well known spin-sum
rules to the indivdual spins of the valence quarks which compose the proton in
our naive valence-model to produce a correct prediction for the protons' spin
${1}\over{2}$. When experimenters set out to measure the contribution of these
valence quarks in 1988 at the EMC experiment~\cite{Ashman1988}, they
were flabbergasted to find that the valence quarks carry only a small fraction
of the proton's spin. Although recent papers~\cite{Povh2016} suggest that this 'spin
crisis' is simple due to misattribution of spin, most literature to date has
focused on understanding how to model the proton with parton distribution
functions. These parton distribution functions come in many varieties, and probe
different degrees of freedom within the proton, in both the case of unpolarized
partion distribution funcitons, and polarized parton distribution functions. \\
 
\section{Scope and Objectives of This Work}
This thesis will describe the research I carried out between May of 2010 through
August of 2016. I will often quote work that was carried out in active
collaboration with Ralf Seidel, Francesca Giordano, Daniel Jumper, Sanghwa Park,
Abraham Meles and Chong Kim. Daniel, Abraham, Ralf, Francesca, and myself all
worked on the 2013 polarized proton data set taken at RHIC with PHENIX. This
analysis comprises the body of work devoted to calculating $A_L$ for the
$W\rightarrow\mu$ decay. Since 2013, the five of us collaborated closely on all
aspects of the work, which provided invaluable cross-checks at nearly every
stage. Many of the figures in this document were produced by our collective
efforts, and I will do my best to cite when possible, if one analyzer played a
particularly large role in generating the data or visualization, however after
several years of working together, I will certainly fail to attribute, or
misattribute at times.

The other portion of this thesis will discuss the Vernier Analysis, which is
instrumental for every single-cross-section calculation taken with RHIC data.
The thrust of the Vernier Analysis is to determine the beam luminosity at
PHENIX's interaction point, so as to normalize these cross-section calculations.
This is done with a series of specialized Vernier-Scans, where beams are scanned
across one-another in order to measure beam geometry. The luminosity can then be
calcualted from first principals, and compared to the advertised machine
luminosity published by RHIC's collider-accelerator department. I began working
with the Vernier Analysis under the tutelage of K. Oleg Eyser, but eventually
moved to work independantly on the analysis, producing an entire software
framework for handling data cleaning, analysis, visualization and simulation.
