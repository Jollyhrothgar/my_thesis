\chapter{Discussion and Conclusion}

The PHENIX results on the Asymmetry put an important constraint on the polarized
parton distribution functions, and spin contributions of the sea-quarks to the
total proton spin. Due to the vanishing asymmetries found in this analysis, we
now know that with: 

\begin{align}
  A_L(x,Q^2) &= {{\sigma^+-\sigma^-}\over{\sigma^++\sigma^-}} \\
        &\equiv {{g_1(x,Q^2)}\over{F_1(x,Q^2)}} \\
        &\approx 0
\end{align}

that the sea quarks must contribute very little to the proton's total spin. We
additionally know, with good constraints on the contributions from the valence
quarks, that the majority of the proton spin must reside in the gluons and the
angular momentum of the proton.

The task of building a RHIC and a PHENIX is truly monumental. The fact that
particle physics can even be done in the first place, is absolutely astounding
to me--the amount of infrastructure, technical expertise, collaboration,
financial and intellectual capital needed to build such an enormous and precise
machine is something that is very difficult to communicate. 

Subsequent experiments such as the new Electron Ion Collider, which will be
completed in coming years will be able to measure these contributions with much
higher precision.
