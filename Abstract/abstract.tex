\begin{abstract}
	This thesis discusses the process of extracting information about the spin
	structure of protons, specifically, spin contributions from the sea of quarks
	and antiquarks, which are kinematically distinct from the 'valence quarks'.
	We have known since the 'proton-spin crisis' ~\cite{Ashman1988} of
	the 1990s that proton spin does not entirely reside in the valence quarks, so
	the thurst of experimental efforts since then have been designed to determine
	both how to probe the proton spin structure, and how to validate models for
	proton spin structure. Here, I discuss one particular approach to
	understanding the sea-quark spin contribution, which utilizes the production
	of real $W$-bosons, and the $W$ coupling with polarized spin structure in the
	proton sea, as produced from polarized protons collisions.  Only one of the
	colliding protons is longitudinally spin polarized, in this analysis, and
	they are collided at an energy of $500 GeV$. The expermental observable used
	is referred to as "$A_L$" which is expressed mathematically as a ratio of
	sums and differences of various helicity combinations of singly polarized
	interactions between two protons, i.e.  $p+p^{\Rightarrow}: \rightarrow W
	\rightarrow \mu + \nu$. Once $A_L$ has been experimentally measured, it can
	then be used to determine appropriate polarizations of proton sea-quarks,
	within a given uncertainty, if we write the cross-sections used in the
	calculation of $A_L$ in terms of polarized parton distribution functions.
	Finally, this thesis will also include a discussion of my work experimentally
	determining the absolute luminosity of collisions at RHIC, which is needed as
	a normalization on any cross section used in the analysis. In particular,
	studying the cross section of the $W$ interaction can help to validate our
	models for assigning a signal-to-background ratio to the $W\rightarrow\mu$
	events.  
\end{abstract}
